%%%%%%%%%%%%%%%%%%%%%%%%%%%%%%%%%%%%%%%%%%%%%%%%%%%%%%%%%%%%%%%%%%%%
%
%   Style for CMS Computing / Physics Technical Design Reports
%
%   Lucas Taylor  4 Feb 2005,   Revised  12 Oct 2005
%
%%%%%%%%%%%%%%%%%%%%%%%%%%%%%%%%%%%%%%%%%%%%%%%%%%%%%%%%%%%%%%%%%%%%

%  the following line is edited by the tdr script to change or to pass
%  additional options:
\documentclass[11pt,twoside,a4paper,an]{cms-tdr}
\def\svnVersion{146739:150530M}

%%%%%%%%%%%%%%%%%%%%%%%%%%%%%%%%%%%%%%%%%%%%%%%%%%%%%%%%%%%%%%%%%%%%

\begin{document}
%%%%%%%%%%%%%%%%%%%%%%%%%%%%%%%%%%%%%%%%%%%%%%%%%%%%%%%%%%%%%%%%%%%%
%
%  Common definitions
%
%  N.B. use of \providecommand rather than \newcommand means
%       that a definition is ignored if already specified
%
%                                              L. Taylor 18 Feb 2005
%%%%%%%%%%%%%%%%%%%%%%%%%%%%%%%%%%%%%%%%%%%%%%%%%%%%%%%%%%%%%%%%%%%%


%%%%%%%%%%%%%%%%%%%%%%%%%%%%%%%%%%%%%%%%%%%%%%%%%%%%%%%%%%%%%%%%%%%%
%
% Hyphenations (only need to add here if you get a nasty word break)
%
\hyphenation{had-ron-i-za-tion}
\hyphenation{cal-or-i-me-ter}
\hyphenation{de-vices}
%
% Hyphenations-end
% % Customizable fields and text areas start with % >> below.
% Lines starting with the comment character (%) are normally removed before release outside the collaboration, but not those comments ending lines

% svn info. These are modified by svn at checkout time.
% The last version of these macros found before the maketitle will be the one on the front page,
% so only the main file is tracked.
% Do not edit by hand!
\RCS$Revision: 150530 $
\RCS$HeadURL: svn+ssh://svn.cern.ch/reps/tdr2/notes/AN-14-XXX/trunk/AN-14-XXX.tex $
\RCS$Id: AN-12-124.tex 150530 2012-10-04 12:46:23Z veelken $
%%%%%%%%%%%%% local definitions %%%%%%%%%%%%%%%%%%%%%
% This allows for switching between one column and two column (cms@external) layouts
% The widths should  be modified for your particular figures. You'll need additional copies if you have more than one standard figure size.
\newlength\cmsFigWidth
\ifthenelse{\boolean{cms@external}}{\setlength\cmsFigWidth{0.85\columnwidth}}{\setlength\cmsFigWidth{0.4\textwidth}}
\ifthenelse{\boolean{cms@external}}{\providecommand{\cmsLeft}{top}}{\providecommand{\cmsLeft}{left}}
\ifthenelse{\boolean{cms@external}}{\providecommand{\cmsRight}{bottom}}{\providecommand{\cmsRight}{right}}

\newcommand{\mtau}{m_{\tau}}
\newcommand{\mnus}{m_{\nu\nu}}
\newcommand{\mvis}{m_{vis}}

\newcommand{\METx}{\ensuremath{E^{\mathrm{miss}}_{x}}\xspace}
\newcommand{\METy}{\ensuremath{E^{\mathrm{miss}}_{y}}\xspace}

\newcommand{\TReg}{\textsuperscript{\textregistered}}
\newcommand{\TCop}{\textsuperscript{\textcopyright}}
\newcommand{\TTra}{\textsuperscript{\texttrademark}}

\newcommand{\avg}[1]{\left< #1 \right>}
\newcommand{\MedMass}{\ensuremath{\avg{{M}_{\tau\tau}}}\xspace}
\newcommand{\MedRatio}{\ensuremath{\MedMass/M_{\tau\tau}^{\mathrm {true}}}\xspace}

\newcommand{\tablesize}{\small}

%\newcommand{\captiontext}{\it}
\newcommand{\captiontext}{}

%%%%%%%%%%%%%%%  Title page %%%%%%%%%%%%%%%%%%%%%%%%
\cmsNoteHeader{AN-14-XXX} % This is over-written in the CMS environment: useful as preprint no. for export versions
% >> Title: please make sure that the non-TeX equivalent is in PDFTitle below
\title{Higgs $\rightarrow \tau\tau$ Matrix Element Primer}

%%%%%%%%%%%%%%% Author list %%%%%%%%%%%%%%%%%%%%%%%%%
% define list of authors
% (plus institute affiliation)

\address[LLR]{
% \\
Laboratoire Leprince-Ringuet, \'Ecole Polytechnique,
91128 Palaiseau,
France 
}

\author[LLR]{L.~Mastrolorenzo}
\author[LLR]{C.~Veelken}



% >> Date
% The date is in yyyy/mm/dd format. Today has been
% redefined to match, but if the date needs to be fixed, please write it in this fashion.
% For papers and PAS, \today is taken as the date the head file (this one) was last modified according to svn: see the RCS Id string above.
% For the final version it is best to "touch" the head file to make sure it has the latest date.
\date{\today}

% >> Abstract
% Abstract processing:
% 1. **DO NOT use \include or \input** to include the abstract: our abstract extractor will not search through other files than this one.
% 2. **DO NOT use %**                  to comment out sections of the abstract: the extractor will still grab those lines (and they won't be comments any longer!).
% 3. **DO NOT use tex macros**         in the abstract: External TeX parsers used on the abstract don't understand them.
\abstract{
  This note represents a working document in which we document our progress
  with applying the Matrix Element method to the Higgs $\rightarrow \tau\tau$ analysis.
}

% >> PDF Metadata
% Do not comment out the following hypersetup lines (metadata). They will disappear in NODRAFT mode and are needed by CDS.
% Also: make sure that the values of the metadata items are sensible and are in plain text (no TeX! -- for \sqrt{s} use sqrt(s) -- this will show with extra quote marks in the draft version but is okay).

\hypersetup{%
pdfauthor={Christian Veelken},%
pdftitle={Higgs $\rightarrow \tau\tau$ Matrix Element Primer},%
pdfsubject={CMS},%
pdfkeywords={CMS, physics}}

\maketitle %maketitle comes after all the front information has been supplied

% >> Text
%%%%%%%%%%%%%%%%%%%%%%%%%%%%%%%%  Begin text %%%%%%%%%%%%%%%%%%%%%%%%%%%%%
%% **DO NOT REMOVE THE BIBLIOGRAPHY** which is located before the appendix.
%% You can take the text between here and the bibiliography as an example which you should replace with the actual text of your document.
%% If you include other TeX files, be sure to use "\input{filename}" rather than "\input filename".
%% The latter works for you, but our parser looks for the braces and will break when uploading the document.
%%%%%%%%%%%%%%%

\section{Matrix elements}
\label{sec:matrix_elements}

In this section we describe the computation of the Matrix Elements
for the Higgs $\rightarrow \tau\tau$ signal and different background processes.

\subsection{Higgs $\rightarrow \tau\tau$ signal}

The matrix element for the signal is given by the expression:
\begin{equation}
w = \frac{1}{\sigma_{acc}} \sum_{a,b} \int dx_{a} dx_{b} \frac{f(x_{a}) f(x_{b})}{x_{a} x_{b} s} \ \delta^2((x_a P_a + x_b P_b) + \sum P_{k}) 
  \vert \mathcal{M} \vert^2 \ W(\vec{y}|\vert\vec{x}) \ d\vec{x} \ \mathcal{R} \ dP_{x}^{recoil} dP_{y}^{recoil} 
\label{eq:meSignal}
\end{equation}
$\sigma_{acc}$ denotes the product of cross--section times acceptance of the analysis for the signal.
The summation $\sum_{a,b}$ extends over gluons plus quark--anti quark pairs.
The symbol $f$ represents the parton distribution functions.
The two--dimensional $\delta$--function imposes conservation of energy and longitudinal momentum between initial and final state.

The phase--space element $d\vec{x}$ represents all particles in the the final state.
The simplest case is the hadronic channel. In this case:
\begin{equation*}
d\vec{x} = \frac{d^3p_{h+}}{(2\pi)^3 2 E_{h+}} \frac{d^3p_{\bar{\nu}}}{(2\pi)^3 2 E_{\bar{\nu}}} \ \frac{d^3p_{h-}}{(2\pi)^3 2 E_{h-}} \frac{d^3p_{\nu}}{(2\pi)^3 2 E_{\nu}},
\end{equation*}
where we by $h+$ ($h-$) the system of hadrons produced in the $\tau^{+}$ ($\tau^{-}$) decay.
We use the recursive relation for phase--space elements given in the kinematics section of the PDG~\cite{PDG} (Eq.{\it 43.12}) to write the phase--space element as:
\begin{eqnarray*}
d\vec{x} & = & (2\pi)^6 \frac{d^3p_{\tau+}}{(2\pi)^3 2 E_{\tau+}} \frac{d^3p_{h+}}{(2\pi)^3 2 E_{h+}} \frac{d^3p_{\bar{\nu}}}{(2\pi)^3 2 E_{\bar{\nu}}}
  \ \delta(E_{\tau+} - E_{h+} - E_{\bar{nu}}) \ \delta^3(\vec{p}_{\tau+} - \vec{p}_{h+} - \vec{p}_{\bar{nu}}) \\
  & & \cdot \frac{d^3p_{\tau+}}{(2\pi)^3 2 E_{\tau+}} \frac{d^3p_{h-}}{(2\pi)^3 2 E_{h-}} \frac{d^3p_{\nu}}{(2\pi)^3 2 E_{\nu}} 
  \ \delta(E_{\tau-} - E_{h-} - E_{\bar{nu}}) \ \delta^3(\vec{p}_{\tau-} - \vec{p}_{h-} - \vec{p}_{\bar{nu}}) dq_{1}^2 dq_{2}^2.   
\end{eqnarray*}
Extending the dimension of the integral may seem like a complication at first, but will simplify the calculus later.

The actual Matrix Element $\vert \mathcal{M} \vert^2$ consists of three parts:
\begin{equation*}
\vert \mathcal{M} \vert^2 = \vert \mathcal{M}_{HS} \vert^2 \cdot \frac{1}{(m_{H}\Gamma_{H})^{2}} \delta(M_{\tau\tau}^2 - m_{H}^2) 
\frac{1}{(m_{\tau}\Gamma_{\tau})^{4}} \delta(q_{1}^2 - m_{\tau}^2) \delta(q_{2}^2 - m_{\tau}^2) \cdot \vert \mathcal{M}_{decay} \vert^2.
\end{equation*}
The first part, $\vert \mathcal{M}_{HS} \vert^2$ denotes the Matrix Element for the ``hard-scatter'' interaction $pp \rightarrow Higgs \rightarrow \tau\tau$, 
which we intend to take from MadGraph~\cite{MadGraph}.
The third part, $\vert \mathcal{M}_{decay} \vert^2$, represents the tau decays.
The second part contains the Breit--Wigner functions of the propagator terms that relate the hard--scatter interaction to the tau decays.

The variables $P_{x}^{recoil}$ and $P_{y}^{recoil}$ represent the hadronic recoil:
\begin{eqnarray*}
P_{x}^{recoil} & = & -\left(\METx + \sum_{e} P_{x}^{e} + \sum_{\mu} P_{x}^{\mu} + \sum_{\tau_{h}} P_{x}^{\tau_{h}} + \sum_{jets} P_{x}^{jet} \right) \\
P_{y}^{recoil} & = & -\left(\METy + \sum_{e} P_{y}^{e} + \sum_{\mu} P_{y}^{\mu} + \sum_{\tau_{h}} P_{y}^{\tau_{h}} + \sum_{jets} P_{y}^{jet} \right) 
\end{eqnarray*}
The issue with the hadronic recoil is that the Matrix Elements generated by MadGraph are leading order.
The system of particles passed to the Matrix Element are expected to be balanced in transvere momentum.
The hadronic recoil spoils this balance, causing the taus produced in the Higgs decay not to be ``back--to--back'' 
and their transverse momenta to extend beyond the kinematic endpoint $P_{T}^{\tau_{h}} = \frac{1}{2} m_{H}$.
We follow Ref.~\cite{Alwall:2010cq} and integrate over the hadronic recoil, using a transfers function $\mathcal{R}$ that constrains the
integration to values compatible with the measured hadronic recoil:
\begin{equation*}
\mathcal{R} = \frac{1}{2\pi \sqrt{\det V}} \exp \left( -\frac{1}{2} \left( \vec{p}^{recoil} - \vec{p}_{exp}^{recoil} \right)^{T} V^{-1} \left( \vec{p}^{recoil} - \vec{p}_{exp}^{recoil} \right) \right),
\end{equation*}
with $\vec{p}_{exp}^{recoil} = -\left( \sum_{taus} \vec{p}_{\tau} \sum_{jets} \vec{p}_{jet} \right)$.
Note that the sum in the expression for $\vec{p}_{exp}^{recoil}$ extend over the ``true'' momenta of the two taus originating from the Higgs decay plus the momenta of high $P_{T}$ jets.

The vector $\vec{p}_{exp}^{recoil}$ can be used to transform the momenta $\vec{p}_{\tau}$ and $\vec{p}_{jet}$ into the {\em zero momentum frame} of the hard-scatter interaction.
The transformation is performed by a Lorentz boost in the transverse plane:
\begin{equation*}
\vec{p}' = \Lambda \vec{p}.
\end{equation*}
The transformation matrix is given by:
\begin{equation*}
\Lambda = \left| \begin{array}{cccc}
 \gamma & -\gamma \beta_{x} & -\gamma \beta_{y} & 0 \\
 -\gamma \beta_{x} & 1 + (\gamma - 1) \frac{\beta_{x}^2}{\beta^2} & (\gamma - 1) \frac{\beta_{x}\beta_{y}}{\beta^2} & 0 \\
 -\gamma \beta_{y} & (\gamma - 1) \frac{\beta_{x}\beta_{y}}{\beta^2} & 1 + (\gamma - 1) \frac{\beta_{y}^2}{\beta^2} & 0 \\
 0 & 0 & 0 & 1 \end{array} \right|,
\end{equation*}
with $\beta_{x} = \frac{P_{x}^{recoil}}{P_{T}^{recoil}}$, $\beta_{y} = \frac{P_{y}^{recoil}}{P_{T}^{recoil}}$,
$P_{T}^{recoil} = \sqrt{{P_{x}^{recoil}}^2 + {P_{y}^{recoil}}^2}$, $\beta = \sqrt{\beta_{x}^2 + \beta_{y}^2}$ and $\gamma = \frac{1}{\sqrt{1 - \beta^2}}$.

\subsubsection{$\vert \mathcal{M}_{HS} \vert^2$}

Christian suggest that we compute $\vert \mathcal{M}_{HS} \vert^2$ separately for Higgs $\rightarrow \tau\tau$ signal events with 0--jet, 1--jet and 2--jet.
Jets are required to have $P_{T} > 30$~\GeV and $\vert \eta \vert < 4.7$ in order to count towards the jet multiplicity.

The Matrix Elements generated by MadGraph can be found in the directory \begin{verbatim}SubProcesses/internal_MadGraph_name_of_the_MatrixElement/matrixXX.f\end{verbatim} .
XX is a number chosen by Madgraph in case different Feynman diagrams contribute to the same final state.
The Feynman diagram for the process is stored in \begin{verbatim}SubProcesses/internal_MadGraph_name_of_the_MatrixElement/matrixXX.ps\end{verbatim} .
The Feynman diagram is neccessary to decide which parton distribution functions (gluon or quark flavor) we need for each Feynman diagram.


\subsubsection{Breit--Wigner terms and tau decays}

The part
\begin{equation*}
T \equiv \frac{1}{(m_{\tau}\Gamma_{\tau})^{2}} \ \delta(q^2 - m_{\tau}^2) 
  \ (2\pi)^3 \ d^3p_{\tau} \ d^3p_{h} \ d^3p_{\nu} \ \delta(E_{\tau} - E_{h} - E_{\nu}) \ \delta^3(\vec{p}_{\tau} - \vec{p}_{h} - \vec{p}_{\bar{\nu}}) \ dq^2
\end{equation*}
has identical structure for each tau.

We first perform the integration over $dq^2$ to get rid of the Breit--Wigner term. 
This yields:
\begin{equation*}
T = (2\pi)^3 \ d^3p_{\tau} \ d^3p_{h} \ d^3p_{\nu} \ \delta(E_{\tau} - E_{h} - E_{\nu}) \ \delta^3(\vec{P}_{\tau} - \vec{p}_{h} - \vec{p}_{\bar{\nu}}).
\end{equation*}
We then use the 3--dimensional $\delta$--function that represents the momentum conservation in the tau decay to perform the integration over $d^3P_{\nu}$:
\begin{equation*}
T = (2\pi)^3 d^3p_{\tau} \ d^3p_{h} \ \delta(E_{\tau} - E_{h} - E_{\nu}(\vec{p}_{\tau}, \vec{p}_{h})). 
\end{equation*}
The energy of the neutrino, $E_{\nu}$, is now a function of the momenta $\vec{p}_{\tau}$ of the tau lepton and $\vec{p}_{h}$ of the hadronic system:
$E_{\nu} = P_{\nu}^2 = (\vec{P}_{\tau} - \vec{P}_{h})^2 = P_{\tau}^2 + P_{h}^2 - 2 P_{\tau} P_{h} \cos\theta$,
where $\theta$ denotes the angle between the $\vec{p}_{\tau}$ and $\vec{p}_{h}$ vectors.
We next write the phase--space element $d^3p_{\tau}$ in polar coordinates ({\it cf.}\ section.~\ref{sec:transformation_to_polar_coordinates}).
We orient the z--axis such that it coincides with the direction of the $\vec{p}_{h}$ vector.
\begin{equation*}
T = (2\pi)^3 \ P_{\tau}^2 \ dP_{\tau} \ d\cos\theta \ d\phi \ d^3p_{h} \ \delta(E_{\tau} - E_{h} - E_{\nu}(\vec{p}_{\tau}, \vec{p}_{h})). 
\end{equation*}
We need to be careful when performing the integration over $P_{\tau}$, as the argument of the $\delta$--function depends on $P_{\tau}$.
Following the rules for $\delta$--functions~\cite{deltaFunctionRules}, $\delta(g(x)) = \sum_{i} \frac{1}{\vert g'(x_i) \vert} \delta(x - x_{i})$,
where the sum extends over all roots $x_{i}$ of the function $g(x)$, we obtain:
\begin{eqnarray}
T & = & (2\pi)^3 \ P_{\tau}^2 \ d\cos\theta \ d\phi \ d^3p_{h} \ \delta(\sqrt{P_{\tau}^2 + m_{\tau}^2} - \sqrt{P_{h}^2 + m_{h}^2} - P_{\tau}^2 - P_{h}^2 + 2 P_{\tau} P_{h} \cos\theta)) \nonumber \\
  & = & (2\pi)^3 \ d\cos\theta \ d\phi d^3p_{h} \ \left( g(x_{+}) + g(x_{-}) \right),
\label{eq:tauBlock}
\end{eqnarray}
with 
\begin{equation*}
g(x) = \frac{x}{P_{\tau}^2 + m_{\tau}^2} - \frac{P_{h} \cos\theta - x}{P_{h}^2 - 2 P_{h} x \cos\theta + x^2}.
\end{equation*}
The two roots $x_{+}$ and $x_{-}$ are given by:
\begin{equation}
x{\pm} = \frac{(m_{\tau}^2 + m_{h}^2) P_{h} \cos\theta \pm \sqrt{(P_{h}^2 + m_{h}^2) \cdot (m_{\tau}^2 - m_{h}^2)^2 - 4 m_{\tau}^2 P_{h}^2 \sin^2\theta}}{4 m_{\tau}^2 P_{h}^2 \sin^2\theta}.
\label{eq:x}
\end{equation}
The remaining integration over $\theta$ in Eq.~\ref{eq:tauBlock} extends over the region for which the expression in the radical of Eq.~\ref{eq:x} remains positive:
\begin{equation*}
(m_{\tau}^2 - m_{h}^2)^2 - 4 m_{\tau}^2 P_{h}^2 \sin^2\theta > 0 \Leftrightarrow 0 < \sin \theta < \frac{m_{\tau}^2 - m_{h}^2}{4 m_{\tau} P_{h}}.
\end{equation*}

\subsubsection{Putting it all together}

Final integration variables:
\begin{itemize}
\item $M_{\tau\tau}$
\item $x_{a}$
\item $\phi_{\tau+}$
\item $\phi_{\tau-}$
\item $P_{h+}$, for energy transfer function of first hadronic tau
\item $P_{h-}$, for energy transfer function of second hadronic tau
\item $P_{x}^{recoil}$
\item $P_{y}^{recoil}$
\end{itemize}

To--do:
\begin{itemize}
\item Perform a variable transformation from $d\theta_{1} d\theta_{2}$ to $dM_{\tau\tau}$, the mass of the tau lepton pair,
  and $u = P_{z_{1}} + P_{z_{2}}$, the sum of longitudinal momenta of the two tau leptons.
  These transformations are neccessary in order to make the integration over the Breit--Wigner term for the Higgs and the integration over $dx_{b}$ numerically stable.
\item Extend the formalism for the tau decays to $\tau \rightarrow e \nu\bar{\nu}$ and $\tau \rightarrow \mu \nu\bar{\nu}$ decays.
\item Check if (LO) $\vert \mathcal{M}_{HS} \vert^2$ obtained from MadGraph can handle finite transverse momentum of the Higgs.
  In case MadGraph requires $P_{T}^{H} = 0$, we need to replace the integration over the hadronic recoil in Eq.~\ref{eq:meSignal} by a two--dimensional $\delta$--function.
\item We need to determine the product of signal cross--section times acceptance, $\sigma_{acc}$, in Eq.~\ref{eq:meSignal}.
  Christian suggest we approximate this by the Standard Model Higgs production cross--section obtained from the LHC Higgs cross section working--group~\cite{LHCHiggsCrossSectionWorkingGroup:2011ti}
  times the signal acceptance as function of mass from the MSSM Higgs $\rightarrow \tau\tau$ analysis~\cite{CMS_AN_2013-171}.
\item We need to find--out in which format the Fortran code for $\vert \mathcal{M}_{HS} \vert^2$ generated by MadGraph expects its input.
  Code for performing the numerical integration and for calling the parton distribution functions already exist in SVfit.
\end{itemize}

For debugging:
\begin{itemize}
\item We appreciate that the matrix element formalism is not trivial and we anticipate that we may make mistakes that we need to debug.
  The idea is to compute Eq.~\ref{eq:meSignal} for a series of test Higgs masses $\{ m_{H}^{i} \}$ in simulated Higgs $\rightarrow \tau\tau$ signal events with $m_{H}^{true} = 125$~\GeV.
  We expect that for most signal events the maximum weight $w$ is attained for a test mass $m_{H}^{i}$ close to the true Higgs mass $m_{H}^{true}$,
  modulo a typical resolution of $\mathcal{O}(20\%)$ that we plan to compare with SVfit~\cite{Bianchini:2014vza}.
\end{itemize}







\section{Acknowledgements}
\label{sec:acknowledgments}

We wish to thank Lorenzo Bianchini for helpful discussions.


%\clearpage

\bibliography{auto_generated}   % will be created by the tdr script.

%\clearpage

\appendix

\section{Collection of formulas}
\label{sec:appendix_collection_of_formulas}

\subsection{Transformations of phase--space element from Cartesian to polar coordinates}

We will assume that the direction of electrons, muons, hadronic taus and jets is measured precisely.

With this assumption the transfer functions that related the measured four--vector \vec{x'} to the true four--vector \vec{x} (used in the Matrix Element):
\begin{equation*}
W(\vec{x'}, \vec{x}) = W(E', E) \delta(\theta' - \theta) \delta(\phi' - \phi).
\end{equation*}

The use of polar coordinates ($P$, $\theta$, $\phi$) or equivalently ($P$, $\eta$, $\phi$) 
is hence advantageous for the computation of transfer functions integrals.

\subsubsection{Transformation of $d^{3}p$ to $dP d\theta d\phi$}
\label{sec:transformation_to_polar_coordinates}

The transformation to ($P$, $\theta$, $\phi$) is given by the expressions:
\begin{eqnarray*}
P_{x} & = & P \cos \phi \sin \theta
P_{y} & = & P \sin \phi \sin \theta
P_{z} & = & P \cos \theta.
\end{eqnarray*}

The Jacobi determinant associated with the transformation is:
\begin{equation*}
\det M = P^2 \sin \theta
\end{equation*}

The phase--space element can hence be written:
\begin{equation*}
\frac{d^3p}{2 E} = \frac{{P^2 \sin\theta \ dP \ d\theta \ d\phi}}{2 E} = \frac{{P^2 \ dP \ d\cos\theta \ d\phi}}{2 E} = \frac{P}{2} \ dE \ d\cos\theta \ d\phi.
\end{equation*}
The last step follows as $E = \sqrt{P^2 + m^2}$ implies $\frac{dE}{dp} = \frac{P}{E} \Leftrightarrow P dP = E dE$.

\subsection{Transformation of $d^{3}p$ to $dP_{T} d\eta d\phi$}

The transformation to ($P_T$, $\eta$, $\phi$) is given by the following expressions:
\begin{eqnarray*}
P_{x} & = & P_{T} \cos \phi
P_{y} & = & P_{T} \sin \phi
P_{z} & = & P_{T} \sinh \eta.
\end{eqnarray*}

The Jacobi determinant associated with this transformation is:
\begin{equation*}
\det M = P_{T}^2 \cosh \eta
\end{equation*}

The phase--space element can hence be written:
\begin{equation*}
\frac{d^3p}{2 E} = \frac{{P_{T}^2 \ \cosh\eta \ dP_{T} \ d\eta \ d\phi}}{2 E}.
\end{equation*}

\subsection{Transformation to resonance mass}

In case the phase--space integral extends over daughter particles of a narrow resonance,
numerical stability of the computation requires to perform a transformation of the integration variables
such that one integration variable corresponds to the mass of the resonance~\cite{Artoisenet:2010cn}.

Starting from the variables ($P_T$, $\eta$, $\phi$) of two daughters, the mass of the resonance can be expressed by the transverse momentum of the second daughter by:
\begin{equation*}
P_{T_{2}} = \frac{m^2}{2 P_{T_{1}} \cos(\phi_{1} - \phi_{2}) + 2 P_{T_{1}} \cosh(\eta_{1} - \eta_{2})}.
\end{equation*}

The Jacobi determinant associated with the transformation is:
\begin{equation*}
\det M = \frac{m}{2 P_{T_{1}} \cos(\phi_{1} - \phi_{2}) + 2 P_{T_{1}} \cosh(\eta_{1} - \eta_{2})}.
\end{equation*}




%%% DO NOT ADD \end{document}!

\end{document}

